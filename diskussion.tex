\section{Diskussion}
\label{sec:Diskussion}

\subsection{Einzelspulen} % (fold)
    Vergleicht man unsere geplotteten Kurven mit den zu erwarteten aus der Theorie sieht man schön den Abfall an Feldstärke mit zunehmender Distanz zur Spule, leider sieht man die Abflachung bzw. die
    Konsitente Feldstärke im inneren der Spulen nur leicht bei der kurzen Spule und bei der langen ist dieses verhalten fehlend. Dies liegt an der geringen Anzahl an Messungen um den Mittelpunkt der Spulen
    und die bei uns gewählten Abstände. Außerdem sieht man gut den Unterschied der Magnetfeldstärke von einer langen zu einer kurzen Spule.
    Im grossen und ganzen fällt das Messergebnis jedoch in Einklang mit der Theorie.
\label{sub:DisEinzelspulen}

% subsection Einzelspulen (end)

\subsection{Helmholtzspule} % (fold)
    Die Helmholtzspulen haben auch einen relativen Einklang mit der Theorie gezeigt, jedoch ist bei den Plots ein unbekannter Faktor von $~2$ aufgetaucht. Dieser Faktor ist entweder durch eine Falsch justierte Sonde
    einen Falsch justierten Amperemeter oder durch einen Rechenfehler zu erklären. Einen einfluss auf die Ergebnisse hat der Faktor nur in hinsicht auf die absolute Amplitude des Feldes, da diese
    aber im Vergleich zur relativen Amplitude eine geringe Aussagekraft hat, kann man dennoch gut die Form der Kurven erkennen. Die Form ist dabei die der Theorie, dies ließe sich gut 
    durch eine Korrektur um den unbekannten Faktor zeigen.
\label{sub:DisHelmholtzspule}

% subsection Helmholtzspule (end)

\subsection{Hysteresekurve} % (fold)
    Durch die hohe Anzahl an Messungen bei der Hysteresekurve lässt sich diese am leichtesten darstellen. Man erkennt gut die bekannte Form einer Hysterekurve und ihrer Neukurve, gleichzeitig kann
    man gut die Werte der Sättigung, Remanenz und Koerzitivkraft ablesen oder wie in unserem Fall mithilfe einer Ausgleichsgerade bestimmen. Problematisch ist unser fehlen der ersten Messwerte um $I=0-1\unit{A}$
    wodurch die Neukurve nicht vollständig darstellen ließ.
\label{sub:DisHysteresekurve}

% subsection Hysteresekurve (end)
