\section{Auswertung}
\label{sec:Auswertung}
\subsection{Einzelspulen}
    \subsubsection{Kurze Spule}
    Die erste Spule die vermessen wurde war eine kurze Spule mit $N=3400$ Windungen, einer Länge von $l=9\unit{cm}$, welche von einem Ende der Spule zum Anderen gemessen wurde, und einen Radius von $R=4.5\unit{cm}$.
    Durch diese Spule wurde eine Stromstärke von $I=0.5\unit{A}$ mit einem Widerstand des Drahtes von $R=62.5\unit{\ohm}$.
    Die Distanzen $\increment x$ sind dabei von dem Ende der Spule gemessen worden, also sind $\increment x < 0$ Werte Innerhalb der Spule. 
    \begin{table}[H]
        \centering
        \begin{tabular}{c c}
            \toprule
            $\increment x$\,/\,$\unit{cm}$& $B$\,/\,$\unit{mT}$\\
            \midrule
            -4.0  &  15\\
            -3.0  &  14\\
            -2.0  &  13\\
            -1.0  &  11\\
            -0.5  &  10\\
            +1.0  &   7\\
            +2.0  &   5\\
            +3.0  &   4\\
            +4.0  &   3\\
            +5.0  &   2\\
            +6.0  &   1\\
            +7.0  &   0\\
            \bottomrule
        \end{tabular}
        \caption{Magnetfeld relativ zur Entfernung vom Spulenende der kurzen Spule.}
        \label{tab:EinzelSpuleTab1}
    \end{table}
    Nun kann man das Magnetfeld gegenüber der Distanz zur Spule in einem Plot auftragen:
    \begin{figure}[H]
        \centering
        \includegraphics{build/EinzelspulePlot1.pdf}
        \caption{Dicke Kurze Spule.}
        \label{fig:EinzelSpulePlot1}
    \end{figure}
    \subsubsection{Lange Spule}
    Die zweite Spule war eine lange Spule mit $N=300$ Windungen, einem Radius $R=2.05\unit{cm}$, die Länge wurde hier leider nicht vermessen.
    Diese Spule wurde beim Daten aufnehmen mit einer Stromstärke von $I=1\unit{A}$ betrieben, dabei hat die Spule einen Widerstand von $R=3.5 \unit{\ohm}$.
    Auch hier sind die Distanzen relativ zum Spulenende angegeben.
    \begin{table}[H]
        \centering
        \begin{tabular}{c c}
            \toprule
            $\increment x$\,/\,$\unit{cm}$& $B$\,/\,$\unit{mT}$\\
            \midrule
            -2.5  &  1.747\\
            -2.0  &  1.449\\
            -1.5  &  1.126\\
            -1.0  &  0.854\\
            -0.5  &  0.649\\
             0    &  0.494\\
            +0.5  &  0.367\\
            +1.0  &  0.249\\
            +2.0  &  0.155\\
            +3.0  &  0.091\\
            +4.0  &  0.090\\
            +5.0  &  0.038\\
            +6.0  &  0.032\\
            +7.0  &  0.013\\
            \bottomrule
        \end{tabular}
        \caption{Magnetfeld relativ zur Entfernung vom Spulenende der langen Spule.}
        \label{tab:EinzelSpuleTab2}
    \end{table}
    \begin{figure}[H]
        \centering
        \includegraphics{build/EinzelspulePlot2.pdf}
        \caption{Dünne Lange Spule.}
        \label{fig:EinzelSpulePlot2}
    \end{figure}

    \label{sec:AuswEinzel}
\subsection{Helmhotzspulenpaar}
    Die Helmhotzspulen haben eine Windungszahl von je $N=100$, einen Radius von $R=6.25\unit{cm}$ und eine Breite von $b=3.3\unit{cm}$.
    Alle folgenden Distanzen in Tabellen sind von der Mitte der linken Spule angegeben, in den Plots jedoch ist dies angepasst und von der Mitte des Helmhotzspulenpaars angegeben.
    Ebenfalls kann man mithilfe der Formel eine theoretische Kurve plotten und mit den Messwerten vergleichen.
    \subsubsection{Spulenabstand von $d=7\unit{cm}$}
    
    \begin{table}[H]
        \centering
        \begin{tabular}{c c}
            \toprule
            $\increment x$\,/\,$\unit{cm}$& $B$\,/\,$\unit{mT}$\\
            \midrule
            2.7   &  1.868\\
            3.2   &  1.861\\
            10.0  &  1.161\\
            11.0  &  0.951\\
            12.0  &  0.773\\
            13.0  &  0.619\\
            14.0  &  0.492\\
            \bottomrule
        \end{tabular}
        \caption{Magnetfeld relativ zur Entfernung vom Spulenende der linken Spule.}
        \label{tab:HelmTab7}
    \end{table}
    \begin{figure}[H]
        \centering
        \includegraphics{build/HelmholtzPlot1.pdf}
        \caption{Helmhotzspulenpaar im Abstand von $7\unit{cm}$.}
        \label{fig:HelmholtzPlot1}
    \end{figure}
    \subsubsection{Spulenabstand von $d=12\unit{cm}$}

    \begin{table}[H]
        \centering
        \begin{tabular}{c c}
            \toprule
            $\increment x$\,/\,$\unit{cm}$& $B$\,/\,$\unit{mT}$\\
            \midrule
            2.7   &  1.327\\
            4.0   &  1.189\\
            6.0   &  1.078\\
            8.0   &  1.148\\
            15.0  &  1.035\\
            18.0  &  0.545\\
            20.0  &  0.348\\
            22.0  &  0.220\\
            \bottomrule
        \end{tabular}
        \caption{Magnetfeld relativ zur Entfernung vom Spulenende der linken Spule.}
        \label{tab:HelmTab12}
    \end{table}
    \begin{figure}[H]
        \centering
        \includegraphics{build/HelmholtzPlot2.pdf}
        \caption{Helmhotzspulenpaar im Abstand von $12\unit{cm}$.}
        \label{fig:HelmholtzPlot2}
    \end{figure}
    \subsubsection{Spulenabstand von $d=18\unit{cm}$}

    \begin{table}[H]
        \centering
        \begin{tabular}{c c}
            \toprule
            $\increment x$\,/\,$\unit{cm}$& $B$\,/\,$\unit{mT}$\\
            \midrule
            2.7   &  1.078\\
            5.0   &  0.758\\
            9.0   &  0.508\\
            12.0  &  0.626\\
            21.0  &  0.916\\
            23.0  &  0.623\\
            25.0  &  0.395\\
            \bottomrule
        \end{tabular}
        \caption{Magnetfeld relativ zur Entfernung vom Spulenende der linken Spule.}
        \label{tab:HelmTab18}
    \end{table}
    \begin{figure}[H]
        \centering
        \includegraphics{build/HelmholtzPlot3.pdf}
        \caption{Helmhotzspulenpaar im Abstand von $18\unit{cm}$.}
        \label{fig:HelmholtzPlot3}
    \end{figure}

    \label{sec:AuswHelm}
\subsection{Hysteresekurve}
    Bei der vermessung der Hysteresekurve wurde eine Toroidspule verwendet mit einem Eisenkern, diese Spule hat eine Windungszahl von $N=595$ und einen Radius von $R=13.5\unit{cm}$.
    Es folgen die Tabellen für die Neukurve und den Rest der Neukurve, also das auf- und abdrehen der Stromstärke.
    \begin{table}[H]
        \centering
        \begin{tabular}{c c}
            \toprule
            $I$\,/\,$\unit{A}$ & $B$\,/\,$\unit{mT}$\\
            \midrule
            0.0  &  16  \\   
            1.0  &  130  \\   
            1.5  &  251  \\   
            2.0  &  304  \\   
            2.5  &  373  \\   
            3.0  &  429  \\   
            3.5  &  476  \\   
            4.0  &  509  \\   
            4.5  &  541  \\   
            5.0  &  566  \\   
            5.5  &  591  \\   
            6.0  &  612  \\   
            6.5  &  632  \\   
            7.0  &  648  \\   
            7.5  &  667  \\   
            8.0  &  681  \\   
            8.5  &  697  \\   
            9.0  &  710  \\   
            9.5  &  723  \\   
            10.0 &  739  \\
            \bottomrule
        \end{tabular}
        \caption{Messdaten der Neukurve.}
        \label{tab:Neukurve}
    \end{table}

    \begin{table}[H]
        \centering
        \begin{tabular}{c c c c}
            \toprule
            $I$\,/\,$\unit{A}$ & $B$\,/\,$\unit{mT}$ &$I$\,/\,$\unit{A}$ & $B$\,/\,$\unit{mT}$\\
            \midrule
            10.0  &  739   &  -10.0  & -702 \\
            9.5   &  729   &  -9.5   & -692 \\
            9.0   &  717   &  -9.0   & -684 \\
            8.5   &  709   &  -8.5   & -674 \\
            8.0   &  699   &  -8.0   & -665 \\
            7.5   &  687   &  -7.5   & -654 \\
            7.0   &  676   &  -7.0   & -643 \\
            6.5   &  664   &  -6.5   & -631 \\
            6.0   &  651   &  -6.0   & -618 \\
            5.5   &  636   &  -5.5   & -604 \\
            5.0   &  621   &  -5.0   & -589 \\
            4.5   &  603   &  -4.5   & -571 \\
            4.0   &  583   &  -4.0   & -550 \\
            3.5   &  559   &  -3.5   & -529 \\
            3.0   &  535   &  -3.0   & -507 \\
            2.5   &  508   &  -2.5   & -476 \\
            2.0   &  473   &  -2.0   & -447 \\
            1.5   &  414   &  -1.5   & -382 \\
            1.0   &  341   &  -1.0   & -310 \\
            0.5   &  254   &  -0.5   & -202 \\
            0.0   &  143   &   0     & -118 \\
            -0.5  &   23   &   0.5   & -1   \\
            -1.0  &  -65   &   1.0   & 84   \\
            -1.5  &  -156  &   1.5   & 174  \\
            -2.0  &  -237  &   2.0   & 260  \\
            -2.5  &  -309  &   2.5   & 341  \\
            -3.0  &  -370  &   3.0   & 396  \\
            -3.5  &  -424  &   3.5   & 447  \\
            -4.0  &  -461  &   4.0   & 488  \\
            -4.5  &  -500  &   4.5   & 525  \\
            -5.0  &  -524  &   5.0   & 554  \\
            -5.5  &  -552  &   5.5   & 579  \\
            -6.0  &  -574  &   6.0   & 598  \\
            -6.5  &  -598  &   6.5   & 621  \\
            -7.0  &  -616  &   7.0   & 638  \\
            -7.5  &  -633  &   7.5   & 653  \\
            -8.0  &  -649  &   8.0   & 668  \\
            -8.5  &  -663  &   8.5   & 684  \\
            -9.0  &  -678  &   9.0   & 698  \\
            -9.5  &  -695  &   9.5   & 711  \\
            -10.0 &  -702  &  10.0   & 724  \\
            \bottomrule
        \end{tabular}
        \caption{Messdaten der Hysteresekurve.}
        \label{tab:Hysteresekurve}
    \end{table}
    Diese Messdaten kann man plotten und anhand des Plots einige Merkmale ablesen.
    Der maximale Wert des Magnetfelds stellt hierbei die Sättigung dar, der y-Achsenabschnitt kann als Remanenz entnommen werden und die Nullstellen sind die Koerzitivkraft: 
    \begin{figure}[H]
        \centering
        \includegraphics{build/HysteresePlot.pdf}
        \caption{Hysteresekurve.}
        \label{fig:HystereseKurve}
    \end{figure}
    Die Sättigung beträgt ca. $B_S=700\unit{mT}$, die Remanenz beträgt ca. $B_R=\pm139.2\unit{mT}$ und die Koerzitivkraft beträgt ca. $H_K=\pm464.3\unit{\frac{A}{m}}$.
    Diese Werte wurden aus den obenstehenden Tabellen entnommen oder mithilfe einer linearen Ausgleichsgerade ermittelt.
    \label{sec:AuswHyst}
