\begin{document}
    \section{Persönlicher Bericht Anna Mendel - Kardiologie}
    Die Kardiologie befasst sich als Teil der inneren Medizin mit der Anatomie des Herzens und Erkrankungen des Herz-
    Kreislauf-Systems.
    Eine der häufigsten Herzerkrankungen sind Herzrhythmusstörungen, dessen physikalische Hintergründe in der Behandlung näher erläutert werden sollen.
    Im Rahmen der Diagnostik gibt es verschiedene Untersuchungsmöglichkeiten zur Erkennung von mangelnder Pumpleistung oder Irregularitäten
    des Herzschlages, welche je nach Art und Ausmaß lebensbedrohlich sein können.
    Auf der Seite der bildgebenden Verfahren ist eine Echokardiographie gängig.
    Diese umfasst eine Ultraschalluntersuchung mit einer Sonde, die basierend auf dem piezoelektrischen Effekt Ultraschallwellen
    im frequenzbereich von 2-20 MHz aussendet und empfängt.
    Bei diesem Effekt wird ein Piezokristall in ein elektrisches Wechselfeld gebracht und somit zum Schwingen angeregt, die
    emittierten Wellen entsprechen dann den Schallwellen und werden je nach Gewebe oder Struktur im Körper absorbiert oder
    mit verschiedenen Schallimpedanzen reflektiert.
    Der reflektierte Anteil wird von der Sonde registriert und in elektrische Signale umgewandelt, die auf einem Bildschirm dargestellt werden.
    Das geschieht in Echtzeit und somit kann der Pumpvorgang des Herzens kontinuierlich untersucht werden und dient zum Beispiel der Beurteilung von
    Herzwanddicken, der Kontraktilität, Insuffizienzen oder Stenosen der Herzklappen, oder auch der Tumorerkennung.
    Das Elektrokardiogramm als weitere diagnostische Methode befasst sich mit der elektrischen Aktivität des Herzens.
    Die Reizweiterleitung der Herzmuskelzellen erfolgt über elektrische Impulse, die gemessen und als Kurve dargestellt werden.
    Die Kurve des EKGs lässt sich in verschiedene Abschnitte einteilen, denen jeweils bestimmte elektrophysiologische Vorgänge im Herzen zugrunde liegen.
    Davon ausgehend können Annomalien in der Kurve analysiert und Aussagen im Hinblick auf Extrasystolen, Herzinfarkte oder ähnliche Problematiken getroffen werden.
    Während der Hospitation auf der kardiologischen Station wurde die Behandlung von Patienten mit Herzrhythmusstörungen mitverfolgt.
    In Folge einer Synkope und aufgetretenden Rhythmusstörungen wurde eine Schrittmacher-Therapie durchgeführt zur Beurteilung
    der Gefährdung durch potentiell wiederholt auftretende Rhythmusstörungen.
    Beim sogenannten Pacing ist der Patient sediert und es werden Defibrillator-Patches angebracht, welche durch einen externen Schrittmacher
    manuell gesteuert werden können.
    \begin{figure}
        \centering
        \caption{Externer Schrittmacher und Elektrokardiogramm.}
        \label{pacing}
        \includegraphics[width=0.6\textwidth]{Schrittmacher}
        \centering
        \end{figure}
    Zudem ist ein Multikanal-EKG angeschlossen, um die Signale interpretieren zu können \ref{pacing}.
    Diese Methode kann verwendet werden, um einem Schema folgend das Herz zu stimulieren, bis Rhythmusstörungen gezielt hervorgerufen werden.
    Ausgehend von einer stetigen Ausgangsfrequenz werden künstliche Extrasystolen in immer kürzer werdenen Abständen erzeugt,
    mithilfe des Elektrokardiogramms wird dann überprüft, bis zu welchen Grenzen die elektrische Reizung von einem Auswurf
    am Herzen gefolgt ist.
    Treten keine Komplikationen auf, so kann von einer geringen Gefährdung des Patienten ausgegangen werden.
    Im Anschluss an die Behandlung wurde eine Katheterablation bei einem Patienten mit persistierendem Vorhofflimmern beobachtet.
    Katheterablationen, als minimalinvasiver Eingriff unter Sedierung, werden als Alternative zu Medikationen eingesetzt,
    da die Störung somit unmittelbar am Herzen beseitigt wird.
    Vorhofflimmern als eigene Krankheit zugrunde liegend sind Herzmuskelfasern, die zum Teil vom linken Vorhof in die Lungenvenen herausragen.
    Diese Muskelbrücken verursachen störende elektrische Signale und der Vorhof kann nicht mehr vernünftig arbeiten.
    Auslöser können aber auch Vorerkrankungen oder Operationen sein.
    Im Falle des beobachteten Patienten lag die Ursache an vorangegangenen Behandlungen und Herzinfarkten, wodurch Herzmuskelzellen
    vernarben und ihre Leitfähigkeit verlieren.
    Da Herzmuskelzellen nicht regenerieren können, bleibt diese Fähigkeit für die betroffenen Zellen auch aus, was bei genügend gesundem Gewebe allerdings nicht weiter problematisch ist.
    Anders verhält es sich mit umliegendem Gewebe oder beeinträchtigten, aber nicht gänzlich kaputten Zellen.
    Diese nehmen zwar noch an der Reizweiterleitung des Herzens Teil, die ausgehenden Signale sind jedoch zu schwach und es kommt zu einer gestörten Innervation.
    Das Ziel der Ablation liegt somit in der vollständigen Vernarbung des Zwittergewebes, sodass die elektrischen Reize am Herzen durch das gesunde Gewebe problemlos geleitet werden können.
    Hierbei gibt es verschiedene Möglichkeiten.
    Bei der Hochfrequenzstrom-Ablation wird ein Katheter, typischerweise über die Leistenvene, zum Herzen geführt und dort wird das Gewebe
    über Strom durch die Katheterspitze zerstört.
    Gängige Werte für ein Zielort sind 25-40 Watt Leistung für eine Dauer von 30-60 Sekunden.
    Als Optimierung für diese Methode dient das High Power Short Duration (HSPD) Verfahren.
    Hierbei fließt der Strom nur für 4-7 Sekunden mit einer Leistung von 70-90 Watt.
    Vorteilhaft ist dabei, dass eine noch fokussiertere Verödung möglcih ist, die Behandlungsdauer kürzer ist und umliegendes Gewebe geschützt wird.
    Dies ist auch das in der begleiteten Behandlung verwendete Verfahren.
    Alternativ zur Ablation mit Strom gibt es die Kryo- oder Kälteablation.
    Hierbei wird ein "Eis-Ballon" über den Katheter vor jede Lungenvene gebracht und das Gewebe für 2-4 Minuten auf -40\textcelsius bis -60\textcelsius runtergekühlt.
    Eine neue Methode bietet die Pulsed Field Ablation (PFA), auch Elektroporation genannt.
    Dabei werden vom Katheter zielgerichtete, ultraschnelle elektrische Felder erzeugt, wodurch mikroskopische Poren in der Herzmuskelzellmembran entstehen.
    Das Herz wird außer Takt gebracht und die Zielzellen sterben ab, angrenzendes Gewebe wird nahezu nicht geschädigt.
    Eine mögliche Art der Bildgebung während der Behandlung stellen Röntgenaufnahmen dar.
    Der Katheter kann im Körper lokalisiert und zielgerichtet geführt werden.
    \begin{figure}
        \centering
        \caption{Geräte CARTO3 Mapping-System.}
        \label{carto}
        \includegraphics[width=0.6\textwidth]{CARTO3}
        \centering
        \end{figure}
    \begin{figure}
        \centering
        \caption{Beispiel eines elektroanatomischen Map.}
        \label{map}
        \includegraphics[width=0.6\textwidth]{3D-Mapping}
        \centering
        \end{figure}
    Um Strahlenaussetzung sowohl für den Patienten, als auch für den behandelnden Arzt zu minimieren, werden auch 3D-Mappingsysteme eingesetzt.
    Die Katheterspitze ist für eine solche Anwendung dahingehend modifiziert, dass drei unterschiedlich ausgerichtete Spulen magnetische Wechselfelder mit verschiedenen Frequenzen aufbauen.
    Sensoren erfassen dann durch die Katheterbewegung im Magnetfeld induzierte Spannungsänderungen und leiten diese Informationen weiter an einen Prozessor,
    welcher anhand der Daten zu jedem Zeitpunkt eine auf 1mm genaue Lokalisation und Orientierung des Katheters im Magnetfeld ermöglicht.
    Der Katheter ist ebenfalls dazu in der Lage die elektrische Leitfähigkeit der abgetasteten Regionen zu erkennen.
    Durch das Abtasten des Herzens wird also ein elektroanatomisches Map konstruiert, welches ein 3D-Modell des Herzens mit farblich codierten Amplituden- und
    Spannungssignale der Herzerregung umfasst.
    So können die Zwittergebiete mit wenig Reizweiterleitung identifiziert und im selben Zuge ablatiert werden.
    Für einen Detailbericht über die Kardiologie habe ich mich entschieden, da ich vor dem Einblick durch das Praktikum nicht mit so vielen
    physikalischen Anwendungsgebieten gerechnet hätte, die über EKGs hinaus gehen.
    Die Thematiken waren äußerst spannend und der Austausch mit dem behandelnden Arzt über die Methoden sind mir nachhaltig im Gedächtnis geblieben.
    

\end{document}
