\section{Fehlerrechnung}
Für im Protokoll aufkommende Fehlerrechnungen werden folgende Formeln verwendet:

Die Formel für die Berechnung eines Mittelwertes:
\begin{equation}
    \bar{x}=\frac{1}{N}\sum^N_{i=1}x_i
    \label{eq:Mittelwert}
\end{equation}

Die Formel für die Berechnung des Fehlers eines Mittelwertes:
\begin{equation}
    \increment\bar{x} = \sqrt{\frac{1}{1-N}\sum^N_{i=1}\left(x_i-\bar{x}\right)^2}  
    \label{eq:MittelwertErr}
\end{equation}

Die Formel für die Berechnung des Fehlers einer allgemeinen Größe:
\begin{equation}
    \increment f=\sqrt{\sum^N_{i=1}\left(\frac{\partial f}{\partial x_i}\right)^2\cdot\left(\increment x_i\right)^2}
    \label{eq:Err}
\end{equation}

Die Formel für die Berechnung der relativen Abweichung einer gemessenen Größe:
\begin{equation}
    \increment x_\text{rel}=\frac{\lvert x_\text{mess}-x_\text{theoretisch}\rvert}{x_\text{theoretisch}}\cdot 100\%
    \label{eq:RelErr}
\end{equation}
\label{sec:Fehlerrechnung}
